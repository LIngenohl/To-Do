\documentclass{article}
\usepackage{enumitem,amssymb}
\newlist{todolist}{itemize}{2}
\setlist[todolist]{label=$\square$}
\usepackage{pifont}
\newcommand{\cmark}{\ding{51}}%
\newcommand{\xmark}{\ding{55}}%
\newcommand{\done}{\rlap{$\square$}{\raisebox{2pt}{\large\hspace{1pt}\cmark}}%
\hspace{-2.5pt}}
\newcommand{\wontfix}{\rlap{$\square$}{\large\hspace{1pt}\xmark}}

\title{To-Do}
\author{Leopold Ingenohl}


\begin{document}
\maketitle


  \begin{todolist}
  \item[\done] Frame the problem
  \end{todolist}

\begin{todolist}

  \item Print Components and Kube Mail 
  \item How can we compare the financial strength of the investor with that of the target?
  \item Literature on signs for a takeover
  \item Motivation of the investor - studies?
  \item Topics: Conglomerate Discount, Asymmetric Information
  \item Read a few SC13D filings of the sample to understand what is going on
  \item Preparation of the final sample (all further analysis will be based on it)


\end{todolist}

\section{Main objective of the paper}

  \begin{center}
    The importance of the investor's financial strength in activist corporate cross-holdings 
  \end{center}

  \subsection{Investors}
    
    \begin{enumerate}

      \item Determine abnormal returns generated by the filing of SC13D. Sample will be corporations only. 

      \item Ranking/grouping of investors by their financial strength/condition. 

      \item Grouping of returns by financial strength of the investor.

      \item Cross-sectional regression of financial strength components with regards to the returns
    \end{enumerate}

  \subsection{Target}

    \begin{enumerate}

      \item Characteristics of the target 
    \end{enumerate}

  \subsection{Investor-Target relation}

    \begin{enumerate}
      \item Industries
      \item Comparison of financial conditions 
      \item Grouping the returns with regards to the above factors
    \end{enumerate}

\section{Preparation fo the base sample}

  \begin{enumerate}
    \item SC 13D \& SC 13D/A Filings? 
    \item Period
    \item Sample of all filings
    \item Filter with 10k reports 
    \item COMPUSTAT \& CRSP Data availability
    \item Manual Selection
  \end{enumerate}

\section{Activism}

  \begin{enumerate}
    \item Since our focus is on portfolio investments, we restrict our sample by cross-referencing the 13D filings with a list of investment managers that have filed a Schedule 13F holdings report at some point in their history. We do this so as not to confuse corporate cross-holdings with \emph{activism from portfolio investors}. This restriction limits our data somewhat, because only institutions holding more than dollar 100 million in US stocks file 13F reports. {Greenwood2009}
    \item No investor activism 
  \end{enumerate}

\section{Expected Return Proxy}

  \begin{enumerate}
    \item The accounting fundamentals used as explanatory variables in the proxies for expected profitability and investment include lagged values of Bt/Mt, a dummy variable for negative earnings, profitability (Yt/Bt) for firms with positive earnings, accruals relative to book
    equity for firms with positive (+ACt/Bt) and negative (ACt/Bt) accruals, investment (dAt/At?1), a dummy variable for firms that do not pay dividends (No Dt) and the ratio of dividends to book equity (Dt/Bt).{Fama2006}

    \item What is the result of the calculation?! 
    \item read paper again 
  \end{enumerate}

\section{Identifying Information}

\begin{enumerate}
  \item GIC variables - will merge them with the current sample - based on the PERMCO and CIK 
  \item Those the right ones? 
  \item NAICS, GIC or SIC
\end{enumerate}

\section{Cross-Section Regression}

\begin{enumerate}
  \item Dummy variables 
  \item year, industry, strength etc. 
\end{enumerate}

\section{title}
\end{document}